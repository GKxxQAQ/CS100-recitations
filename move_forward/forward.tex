\section{Perfect Forwarding}

\subsection{Examples}

\begin{frame}[fragile]{Forward Arguments}
  Some functions need to forward the arguments to another function.
  \begin{cpp}
std::invoke(f, x, y, z);
std::vector<Type> v;
v.emplace_back(x, y, z);
  \end{cpp}
  \begin{itemize}
    \item \boxilcpp{std::invoke(f, args...)} calls \boxilcpp{f(args...)}.
    \item \boxilcpp{v.emplace_back(args...)} constructs the element by calling the constructor \boxilcpp{Type(args...)}.
    \begin{itemize}
      \item It is different from \boxilcpp{v.push_back(Type(args...))} in that it does not copy the object.
    \end{itemize}
  \end{itemize}
\end{frame}

\begin{frame}[fragile]{Forward Arguments}
  \blue{Value categories} must be preserved.
  \begin{cpp}[\small]
std::vector<std::string> vs;
std::string s = some_value();
vs.emplace_back(s);            // copy
vs.emplace_back(std::move(s)); // move
  \end{cpp}
  \pause
  \blue{cv-qualifiers} must be preserved.
  \begin{cpp}[\small]
struct Widget {
  Widget(const std::string &);
  Widget(std::string &);
};
std::vector<Widget> vw;
vw.emplace_back(s);     // Widget(std::string &)
vw.emplace_back("abc"); // Widget(const std::string &)
  \end{cpp}
  \pause
  Still, we need to avoid unnecessary copies.
\end{frame}

\subsection{Universal Reference}

\begin{frame}[fragile]{Universal Reference}
  \begin{cpp}
template <typename T>
void fun(T &&x) {
  std::cout << gkxx::get_type_name<T>() << '\n';
}
int main() {
  int i = 42;
  fun(i);
  fun(42);
  const int j = 42;
  fun(j);
  return 0;
}
  \end{cpp}
  Output:
  \begin{txt}
int&
int
const int&
  \end{txt}
\end{frame}

\begin{frame}[fragile]{Universal Reference}
  \begin{cpp}
template <typename T>
void fun(T &&x);
  \end{cpp}
  \begin{itemize}
    \item If the argument is an \blue{rvalue} of type \boxilcpp{Tp}, this is a normal rvalue reference initialization and \ilcpp{T = Tp}.
    \item If the argument is an \blue{lvalue} of type \boxilcpp{Tp}, it follows the special rule:
    \begin{itemize}
      \item \boxilcpp{T} would be deduced to an lvalue reference, i.e. \ilcpp{T = Tp &}.
      \item \boxilcpp{x} is of type \boxilcpp{Tp & &&}, which \textit{collapses} to \boxilcpp{Tp &} through the \textbf{reference collapsing rule}.
    \end{itemize}
    \pause
    \item cv-qualifiers would be preserved since this is a reference initialization.
    \begin{itemize}
      \item e.g. \boxilcpp{x} will be an lvalue reference-to-\ilcpp{const} if the argument is a \ilcpp{const} lvalue.
    \end{itemize}
  \end{itemize}
\end{frame}

\begin{frame}[fragile]{Reference Collapsing}
  It is permitted to form ``references to references'' through type manipulations in \textbf{templates} or \textbf{type aliasing}, in which case the \textit{reference collapsing} rules apply:
  \begin{itemize}
    \item \boxilcpp{& &}, \boxilcpp{& &&} and \boxilcpp{&& &} collapse to \boxilcpp{&}.
    \item \boxilcpp{&& &&} collapses to \boxilcpp{&&}.
  \end{itemize}
  \begin{cpp}
using lref = int &;
using rref = int &&;
int n = 42;
 
lref&  r1 = n;      // int&
lref&& r2 = n;      // int&
rref&  r3 = n;      // int&
rref&& r4 = 42;     // int&&
  \end{cpp}
\end{frame}

\begin{frame}[fragile]{Universal Reference vs Rvalue Reference}
  The form \boxilcpp{T &&x} is a universal reference \textbf{iff} \boxilcpp{T} is obtained through \textbf{type deduction} directly.
  \begin{cpp}
template <typename T>
void fun(T &&x);            // universal reference

template <typename T>
class Widget {
  void fun(T &&x);          // rvalue reference
};

auto &&x = y;               // universal reference

template <typename T>
void f(std::vector<T> &&x); // rvalue reference
  \end{cpp}
\end{frame}

\subsection{\ttt{std::forward}}

\begin{frame}[fragile]{\ttt{std::forward}}
  Defined in header file \ttt{<utility>}.
  \begin{cpp}
template <typename Func, typename T>
auto invoke1(Func f, T &&arg) {
  return f(std::forward<T>(arg));
}
  \end{cpp}
  \begin{itemize}
    \item \boxilcpp{std::forward<T&>(x)} returns an lvalue reference, which produces an lvalue.
    \item \boxilcpp{std::forward<T>(x)}, where \boxilcpp{T} is not a reference, returns an rvalue reference, which produces an rvalue.
    \begin{itemize}
      \item In this case, it is equivalent to \boxilcpp{std::move(x)}.
    \end{itemize}
  \end{itemize}
\end{frame}

\begin{frame}[fragile]{\ttt{std::forward}}
  \ilcpp{std::forward} \textbf{does not actually ``forward'' anything!} It is used to preserve all the details about an argument's type (including value categories and cv-qualifiers).
  \pause
  \par Combining \ilcpp{std::forward} with \blue{universal reference} and \blue{variadic template}, we can perfectly forward any number of arguments with any types:
  \begin{cpp}
template <typename Func, typename... Args>
auto invoke(Func f, Args &&...args) {
  return f(std::forward<Args>(args)...);
}
  \end{cpp}
\end{frame}

\begin{frame}[fragile]{Example: \ttt{std::invoke}}
  To avoid copying the callable object, we may forward \boxilcpp{f} as well:
  \begin{cpp}
template <typename Func, typename... Args>
auto invoke(Func &&f, Args &&...args) {
  return (std::forward<Func>(f))(
    std::forward<Args>(args)...
  );
}
  \end{cpp}
  \begin{itemize}
    \item[*] The return-type might be problematic here...
  \end{itemize}
\end{frame}

\begin{frame}[fragile]{Example: \ttt{vector<T>::emplace\_back}}
  \begin{cpp}[\scriptsize]
template <typename T, typename Alloc>
class vector {
 public:
  template <typename... Args>
  void emplace_back(Args &&...args) {
    check_and_realloc();
    using all_tr = std::allocator_traits<Alloc>;
    all_tr::construct(s_alloc, m_data + m_size,
                      std::forward<Args>(args)...);
    ++m_size;
  }
  void push_back(value_type &&x) {
    emplace_back(std::move(x)); // move
  }
  void push_back(const value_type &x) {
    emplace_back(x);            // copy
  }
};
  \end{cpp}
\end{frame}

\begin{frame}[fragile]{Example: a Python-style \ttt{print}}
  \begin{cpp}
template <typename First, typename... Rest>
inline void print(First &&first, Rest &&...rest) {
  std::cout << std::forward<First>(first);
  if constexpr (sizeof...(rest) == 0)
    std::cout << '\n';
  else {
    std::cout << ' ';
    print(std::forward<Rest>(rest)...);
  }
}
inline void print() {
  std::cout << '\n';
}
  \end{cpp}
\end{frame}

\begin{frame}{Reading Materials}
  \textit{Effective Modern C++}:
  \begin{itemize}
    \item Item 26: Avoid overloading on universal references.
    \item Item 27: Familiarize yourself with alternatives to overloading on universal references.
    \item Item 29: Assume that move operations are not present, not cheap, and not used.
    \item Item 30: Familiarize yourself with perfect forwarding failure cases.
    \item Item 41: Consider pass by value for copyable parameters that are cheap to move and always copied.
  \end{itemize}
\end{frame}