\section{Move Semantics}

\begin{frame}[fragile]{A special constructor/\bluett{operator}\ttt{=}?}
  We need a special constructor/\ilcpp{operator=} that
  \begin{itemize}
    \item is different than copy operations, and
    \item has the semantics of ``taking ownership of resources''.
  \end{itemize}
  What would the parameter type be?
\end{frame}

\subsection{Rvalue Reference}

\begin{frame}[fragile]{Rvalue Reference}
  A kind of reference that is bound to rvalues:
  \begin{cpp}
int &r = 42;            // Error.
int &&rr = 42;          // Correct.
const int &cr = 42;     // Also correct.
const int &&crr = 42;   // Correct but useless.
int i = 42;
int &&rr2 = i;          // Error.
int &r2 = i * 42;       // Error.
const int &cr2 = i * 2; // Correct.
int &&r3 = i * 42;      // Correct.
  \end{cpp}
  \pause
  \begin{itemize}
    \item (Lvalue) references can only be bound to lvalues.
    \item Rvalue references can only be bound to rvalues.
    \item (Lvalue) reference-to-\const can also be bound to rvalues.
    \item Rvalue reference-to-\const is useless in most cases (we will see why).
  \end{itemize}
\end{frame}

\begin{frame}[fragile]{Overload Resolution for References}
  \begin{cpp}
void fun(const std::string &);
void fun(std::string &&);
  \end{cpp}
  \begin{itemize}
    \item \ilcpp{fun("hello")} matches \ilcpp{fun(std::string &&)}.
    \item \ilcpp{fun(s)} matches \ilcpp{fun(const std::string &)}.
    \item \ilcpp{fun(s1 + s2)} matches \ilcpp{fun(std::string &&)}.
    \begin{itemize}
      \item But if \ilcpp{fun(std::string &&)} is not present, calls with rvalue arguments also match \ilcpp{fun(const std::string &)}.
    \end{itemize}
  \end{itemize}
\end{frame}

\begin{frame}[fragile]{Overload Resolution for References}
  \begin{cpp}
void fun(int);
void fun(int &&);
  \end{cpp}
  \begin{itemize}
    \item \ilcpp{fun(i)} matches \ilcpp{fun(int)}.
    \item \ilcpp{fun(42)} is \textbf{ambiguous} (compile-error).
  \end{itemize}
  \pause
  \begin{cpp}
void test(int);
void test(int &);
  \end{cpp}
  \begin{itemize}
    \item \ilcpp{test(42)} matches \ilcpp{test(int)}.
    \item \ilcpp{test(i)} is \textbf{ambiguous}.
  \end{itemize}
\end{frame}

\subsection{Move Operations}

\begin{frame}[fragile]{Overview}
  The \blue{move constructor} and the \blue{move assignment operator}.
  \begin{cpp}
class Widget {
 public:
  Widget(Widget &&) noexcept;
  Widget &operator=(Widget &&) noexcept;
};
  \end{cpp}
  \begin{itemize}
    \item Move operations should be \ilcpp{noexcept} in most cases (we will see this later).
  \end{itemize}
\end{frame}

\begin{frame}[fragile]{The Move Constructor}
  The resources owned by the ``moved-from'' object are \textit{stolen} in move operations.
  \begin{cpp}
template <typename T>
class Array {
  std::size_t m_size;
  T *m_data;
 public:
  Array(Array &&other) noexcept
      : m_size(other.m_size), m_data(other.m_data) {
    other.m_size = 0;
    other.m_data = nullptr;
  }
};
  \end{cpp}
\end{frame}

\begin{frame}[fragile]{The Move Assignment Operator}
  \begin{cpp}
template <typename T>
class Array {
  std::size_t m_size;
  T *m_data;
 public:
  Array &operator=(Array &&other) noexcept {
    if (this != &other) {
      delete[] m_data;
      m_size = other.m_size;
      m_data = other.m_data;
      other.m_size = 0;
      other.m_data = nullptr;
    }
    return *this;
  }
};
  \end{cpp}
\end{frame}

\begin{frame}[fragile]{The Move Constructor}
  \begin{cpp}
template <typename T>
Array<T>::Array(Array &&other) noexcept
    : m_size(other.m_size), m_data(other.m_data) {
  @\onslide<2->@other.m_size = 0;
  @\onslide<2->@other.m_data = nullptr;
@\onslide<1->@}
  \end{cpp}
  \begin{itemize}
    \item Obtain the resources directly instead of making a copy.
    \onslide<2->\item Make sure the ``moved-from'' object is in a valid state and can be safely destroyed.
  \end{itemize}
\end{frame}

\begin{frame}[fragile]{The Move-Assignment Operator}
  \begin{cpp}
template <typename T>
Array<T> &Array<T>::operator=(Array &&other) noexcept {
  @\onslide<1->@if (this != &other) {
    @\onslide<2->@delete[] m_data;
    @\onslide<2->@m_size = other.m_size;
    @\onslide<2->@m_data = other.m_data;
    @\onslide<3->@other.m_size = 0;
    @\onslide<3->@other.m_data = nullptr;
  @\onslide<1->@}
}
  \end{cpp}
  \begin{itemize}
    \item Test self-assignment directly.
    \onslide<2->\item Obtain the resources.
    \onslide<3->\item Make sure the ``moved-from'' object is in a valid state and can be safely destroyed.
  \end{itemize}
\end{frame}

\begin{frame}[fragile]{Copy-and-Swap Still Works!}
  \begin{cpp}
template <typename T>
class Array {
 public:
  void swap(Array &other) noexcept {
    using std::swap;
    swap(m_size, other.m_size);
    swap(m_data, other.m_data);
  }
  Array &operator=@\pinkbox[6.5em]@(Array other) noexcept {
    Array(other).swap(*this);
    return *this;
  }
};
  \end{cpp}
  \begin{itemize}
    \item Surprisingly, we obtain both a copy-assignment operator and a move-assignment operator!
  \end{itemize}
\end{frame}

\begin{frame}[fragile]{Lvalues are Copied; Rvalues are Moved}
  Lvalues are copied; rvalues are moved...
  \begin{cpp}
Array<int> arr = some_value();
Array<int> arr2 = arr; // copy
Array<int> arr3 = arr.slice(l, r); // move
  \end{cpp}
  \pause
  ... but rvalues are copied if there is no move constructor.
  \begin{cpp}
struct Widget {
  Widget(Widget &&) = delete;
  Widget(const Widget &) = default;
};
Widget f();
Widget w = f(); // copy (before C++17)
  \end{cpp}
\end{frame}

\begin{frame}[fragile]{Call Move Operations}
  \begin{cpp}
class Widget {
  Array<int> m_array;
  std::string m_str;
 public:
  Widget(Widget &&other) noexcept
      : m_array(other.m_array), m_str(other.m_str) {}
};
  \end{cpp}
  \pause
  Unfortunately, this will call the \textbf{copy constructors} instead of move constructors.
  \begin{question}
    Is rvalue reference an lvalue or an rvalue?
  \end{question}
\end{frame}

\begin{frame}[fragile]{Lvalues Persist; Rvalues are Ephemeral}
  Roughly speaking,
  \begin{itemize}
    \item \blue{lvalues} have persistent state, whereas
    \item \blue{rvalues} are often \textbf{literals} or \textbf{temporary objects} that only live within an expression.
    \begin{itemize}
      \item Rvalues are about to be destroyed and won't be used by anyone else.
    \end{itemize}
  \end{itemize}
  \pause
  By referring to an rvalue, an rvalue reference is \textbf{extending} the lifetime of it.
  \begin{itemize}
    \item Lvalue reference-to-\ilcpp{const} also does this.
    \item An rvalue reference is an \textbf{lvalue} because it has persistent state.
  \end{itemize}
\end{frame}

\subsection{\ttt{std::move}}

\begin{frame}[fragile]{Generate an Rvalue}
  By casting to an rvalue reference using \ilcpp{static_cast}, we can produce an rvalue manually:
  \begin{cpp}
std::string s(t); // copy
std::string s2(static_cast<std::string &&>(t)); // move
  \end{cpp}
  \pause
  The standard library function \ilcpp{std::move} does this.
  \begin{cpp}
std::string s3(std::move(s)); // move
  \end{cpp}
  Note: a function call whose return type is rvalue reference to object is treated as an rvalue.
\end{frame}

\begin{frame}[fragile]{\ttt{std::move}}
  Defined in header \ttt{<utility>}.
  \begin{itemize}
    \item \ilcpp{std::move} performs a \ilcpp{static_cast} to rvalue reference, which produces an rvalue.
    \item \ilcpp{std::move} is used to \textit{indicate} that an object may be ``moved from''.
    \begin{itemize}
      \item \textbf{It does not ``move'' anything in fact!}
    \end{itemize}
  \end{itemize}
  \pause
  Possible implementation:
  \begin{cpp}
template <typename T>
[[nodiscard]] constexpr auto move@\pinkbox[3.5em]@(T &&t) noexcept
    -> std::remove_reference_t<T> && {
  return static_cast<std::remove_reference_t<T> &&>(t);
}
  \end{cpp}
  \begin{itemize}
    \item[*]{\small The parameter is a \textbf{universal reference}, which we will talk about later.}
  \end{itemize}
\end{frame}

\begin{frame}[fragile]{Call Move Operations}
  \begin{cpp}
class Widget {
  Array<int> m_array;
  std::string m_str;
 public:
  Widget(Widget &&other) noexcept
      : m_array(std::move(other.m_array)),
        m_str(std::move(other.m_str)) {}
  Widget &operator=(Widget &&other) noexcept {
    m_array = std::move(other.m_array);
    m_str = std::move(other.m_str);
    return *this;
  }
};
  \end{cpp}
\end{frame}

\begin{frame}[fragile]{The Moved-from Object}
  What might be the output?
  \begin{cpp}
int i = 42;
int j = std::move(i);
std::cout << i << '\n';
std::string s = "hello";
std::string t = std::move(s);
std::cout << s << '\n';
  \end{cpp}
  \pause
  \begin{itemize}
    \item After a move operation, the moved-from object remains a valid, destructible object,
    \item but users may make no assumptions about its value.
    \pause
    \item The moved-from object is possibly modified in a move operation.
    \begin{itemize}
      \item That's why rvalue reference-to-\ilcpp{const} is rarely used.
    \end{itemize}
  \end{itemize}
\end{frame}

\subsection{The Rule of Five}

\begin{frame}[fragile]{Synthesized Move Operations}
  \begin{cpp}
class Widget {
  Array<int> m_array;
  std::string m_str;
 public:
  Widget(Widget &&) = default;
  Widget &operator=(Widget &&) = default;
}
  \end{cpp}
  \begin{itemize}
    \item The synthesized move operations call the corresponding move operations of each member in the order in which they are declared.
    \item The synthesized move operations are \ilcpp{noexcept}.
  \end{itemize}
\end{frame}

\begin{frame}[fragile]{The Rule of Five}
  The updated copy control members:
  \begin{itemize}
    \item Copy constructor
    \item Copy-assignment operator
    \item Move constructor
    \item Move-assignment operator
    \item Destructor
  \end{itemize}
  \pause
  If one of them is user-declared, the copy control of the class is thought of to have special behaviors.
  \begin{itemize}
    \item Therefore, the move ctor or move-assignment operator will not be generated if any of the rest four members has been declared by the user.
  \end{itemize}
\end{frame}

\begin{frame}{The Rule of Five}
  \begin{itemize}
    \item The move ctor or move-assignment operator will not be generated if any of the rest four members has been declared by the user.
    \item The copy ctor or copy-assignment operator, if not provided by the user, will be implicitly \ilcpp{delete}d if the class has a user-declared move operation.
    \item The generation of the copy ctor or copy-assignment operator is \textbf{deprecated} {\small\teal{(since C++11)}} when the class has a user-declared copy operation or destructor.
  \end{itemize}
  To sum up, the five copy control members are thought of as a unit in modern C++: \textbf{If you think it necessary to define one of them, consider defining them all.}
\end{frame}

\subsection{Move Operations and Exceptions}

\begin{frame}[fragile]{Move Operations and Exceptions}
  Consider how \ilcpp{std::vector} grows:
  \begin{cpp}[\scriptsize]
template <typename T, typename Alloc>
void vector<T, Alloc>::reallocate(size_type cap) {
  using all_tr = std::allocator_traits<Alloc>;
  auto new_data = all_tr::allocate(s_alloc, cap), p = new_data;
  for (size_type i = 0; i != m_size; ++i, ++p)
    all_tr::construct(s_alloc, p, m_data[i]);
  m_free(); // destroys all elements and deallocates memory
  m_data = new_data;
  m_capacity = cap;
}
  \end{cpp}
\end{frame}

\begin{frame}[fragile]{Move Operations and Exceptions}
  To enable \blue{strong exception safety guarantee}:
  \begin{cpp}[\scriptsize]
template <typename T, typename Alloc>
void vector<T, Alloc>::reallocate(size_type cap) {
  using all_tr = std::allocator_traits<Alloc>;
  auto new_data = all_tr::allocate(s_alloc, cap), p = new_data;
  try {
    for (size_type i = 0; i != m_size; ++i, ++p)
      all_tr::construct(s_alloc, p, m_data[i]);
  } catch (...) {
    while (p != new_data)
      all_tr::destroy(s_alloc, --p);
    all_tr::deallocate(s_alloc, new_data, cap);
    throw;
  }
  m_free();
  m_data = new_data;
  m_capacity = cap;
}
  \end{cpp}
\end{frame}

\begin{frame}[fragile]{Move Operations and Exceptions}
  With C++11, a natural optimization is to move-construct each element when \ilcpp{value_type} is move-constructible:
  \begin{cpp}[\scriptsize]
template <typename T, typename Alloc>
void vector<T, Alloc>::reallocate(size_type cap) {
  using all_tr = std::allocator_traits<Alloc>;
  auto new_data = all_tr::allocate(s_alloc, cap), p = new_data;
  try {
    for (size_type i = 0; i != m_size; ++i, ++p)
      all_tr::construct(s_alloc, p, @\scriptsizepinkbox[10em]@std::move(m_data[i]));
  } catch (...) {
    while (p != new_data)
      all_tr::destroy(s_alloc, --p);
    all_tr::deallocate(s_alloc, new_data, cap);
    throw;
  }
  m_free();
  m_data = new_data;
  m_capacity = cap;
}
  \end{cpp}
\end{frame}

\begin{frame}[fragile]{Move Operations and Exceptions}
  What if the move constructor throws an exception?
  \begin{cpp}[\scriptsize]
  try {
    for (size_type i = 0; i != m_size; ++i, ++p)
@\scriptsizepinkbox[0.9\textwidth]\quad\danger @  all_tr::construct(s_alloc, p, std::move(m_data[i]));
  } catch (...) {
    while (p != new_data)
      all_tr::destroy(s_alloc, --p);
    all_tr::deallocate(s_alloc, new_data, cap);
    throw;
  }
  \end{cpp}
  \pause
  The preceding elements have been moved! How can we restore them?
\end{frame}

\begin{frame}[fragile]{Move Operations and Exceptions}
  Exception is not welcome in move operations.
  \begin{itemize}
    \item Copy is to \textit{create something else} in terms of existing things,
    \item whereas move is to \textit{change} the existing things.
  \end{itemize}
  \pause
  Use \ilcpp{std::move_if_noexcept} to move the elements only when the move constructor does not throw.
  \begin{cpp}[\small]
for (size_type i = 0; i != m_size; ++i, ++p)
  all_tr::construct(s_alloc, p,
                    std::move_if_noexcept(m_data[i]));
  \end{cpp}
\end{frame}

\begin{frame}[fragile]{\ttt{std::move\_if\_noexcept}}
  Possible implementation:
  \begin{cpp}
template <typename T>
[[nodiscard]] constexpr std::conditional_t<
  !std::is_nothrow_move_constructible_v<T>
    && std::is_copy_constructible_v<T>,
  const T &,
  T &&
> move_if_noexcept(T &&x) noexcept {
  return std::move(x);
}
  \end{cpp}
  Note: for move-only types (for which copy constructor is not available), move constructor is used either way and the strong exception-safety guarantee may be waived.
\end{frame}

\subsection{Automatic Move and Copy Elision}

\begin{frame}{Move from Local Variables and Parameters}
  When an object {\footnotesize(non-\bluett{volatile})} being \ilcpp{return}ed is declared
  \begin{itemize}
    \item in the function body, or
    \item as a parameter of the function,
  \end{itemize}
  the \blue{overload resolution} to select the constructor to use for initialization of the returned value is performed twice:
  \begin{enumerate}
    \item first as if the object were an \textbf{rvalue} (thus it may select the move constructor),
    \item and if the first overload resolution failed, then overload resolution is performed as usual, with the object considered as an lvalue (so it may select the copy constructor).
  \end{enumerate}
\end{frame}

\begin{frame}[fragile]{Move from Local Variables and Parameters}
  In short, the returned value will be copy-initialized only when the move constructor is not available:
  \begin{cpp}
std::string foo() {
  std::string s = some_value();
  return s;
}
std::string t = foo();
  \end{cpp}
  This causes only two moves.
\end{frame}

\begin{frame}[fragile]{Guaranteed Copy Elision}
  Since C++17, elision of copy/move operations are mandatory (instead of a compiler optimization) in some cases, e.g.
  \begin{itemize}
    \item \bluett{return}ing a \blue{prvalue} of the same class type (ignoring cv-qualification) as the function return-type:
    \begin{cpp}
std::string f() {
  return std::string(10, 'c');
}
f(); // only calls std::string(10, 'c')
     // No copy or move is made.
    \end{cpp}
    \item Initializing an object with a \blue{prvalue} initializer of the same class type (ignoring cv-qualification):
    \begin{cpp}
std::string s = f(); // No copy or move is made.
// equivalent to `std::string s(10, 'c');'
    \end{cpp}
  \end{itemize}
\end{frame}

\begin{frame}[fragile]{Guaranteed Copy Elision}
  With C++17 \blue{copy elision}, the following code causes only one move:
  \begin{cpp}
std::string foo() {
  std::string s = some_value();
  return s;
}
std::string t = foo();
  \end{cpp}
  \pause
  This code compiles even with \boxilcpp{Widget(Widget &&) = delete;}:
  \begin{cpp}
Widget fun() {
  return Widget{};
}
Widget w = fun(); // No copy or move is made.
  \end{cpp}
\end{frame}