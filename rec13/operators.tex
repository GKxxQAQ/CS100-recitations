\section{Operator Overloading}

\subsection{Function-Call Operator}

\begin{frame}[fragile]{Function-Call Operator}
    \begin{cpp}
inline int adder(int a, int b) {
  return a + b;
}
int x = adder(42, 35);
    \end{cpp}
    The expression `\ttt{adder(42, 35)}' consists of one operator and \textbf{three} operands:
    \begin{itemize}
        \item \ttt{()} is the \textbf{function-call} operator.
        \item Three operands are \ttt{adder}, \ttt{42} and \ttt{35}.
    \end{itemize}
\end{frame}

\begin{frame}[fragile]{Function Objects (Functors)}
    \begin{cpp}
struct Adder {
  int operator()(int a, int b) const {
    return a + b;
  }
};
int x = Adder{}(42, 35);
int y = Adder()(42, 35); // Equivalent
    \end{cpp}
    \textbf{Default-construct} an \ttt{Adder} object first, and then call \ttt{operator()}.
    \begin{itemize}
        \item \textbf{Function objects} or \textbf{Functors}: Objects of classes that define the call operator.
        \item ``Act like functions''.
    \end{itemize}
\end{frame}

\begin{frame}[fragile]{Functors}
    \begin{cpp}
struct Add_k {
  int k;
  Add_k(int x) : k(x) {}
  void operator()(int &x) const {
    x += k;
  }
};
int x = 42;
Add_k{5}(x); // x becomes 47.
std::vector<int> v = some_value();
std::for_each(v.begin(), v.end(), Add_k{10});
    \end{cpp}
\end{frame}

\subsection{Arithmetic and Relational Operators}

\begin{frame}[fragile]{Arithmetic and Relational Operators}
    \begin{cpp}
class Rational {
 public:
  Rational(int x) : m_num(x), m_denom(1) {}
  Rational() : Rational(0) {}
  double to_double() const {
    return (double)m_num / m_denom;
  }
  // other members
};
inline bool operator==
    (const Rational &lhs, const Rational &rhs) {
  return
    std::fabs(lhs.to_double()-rhs.to_double()) < 1e-9;
}
    \end{cpp}
\end{frame}

\begin{frame}[fragile]{Arithmetic and Relational Operators}
    \begin{cpp}
inline bool operator!=
    (const Rational &lhs, const Rational &rhs) {
  return !(lhs == rhs); // Let operator== do the work.
}
    \end{cpp}
    \pause
    Do we define them as members or non-members?
    \pause
    \begin{cpp}
Rational r = some_value();
if (r == 0)
  // do something.
    \end{cpp}
\end{frame}

\begin{frame}[fragile]{Member or Non-member?}
    \begin{cpp}
class Rational {
 public:
  bool operator==(const Rational &rhs) const {
    // ...
  }
  // other members
};
Rational r = some_value();
if (0 == r) // ERROR!
  // do something.
    \end{cpp}
    \pause
    \textit{Effective C++} Item 24: Declare non-member functions when type conversions should apply to all parameters.
\end{frame}

\subsection{Increment and Decrement Operators}

\begin{frame}[fragile]{Prefix and Postfix}
    Increment/decrement operators are preferred to be members, though not required by the standard.
    \begin{columns}
        \begin{column}{0.5\linewidth}
            \begin{cpp}
class Rational {
 public:
  Rational &operator++() {
    m_num += m_denom;
    return *this;
  }
  Rational operator++(int) {
    auto tmp = *this;
    ++*this;
    return tmp;
  }
};
            \end{cpp}
        \end{column}
        \begin{column}{0.5\linewidth}
            \begin{itemize}
                \item Prefix \bluett{operator++} returns reference to \ttt{*}\bluett{this},
                \item while postfix \bluett{operator++} returns copy of the object before incrementation.
                \item Postfix \bluett{operator++} has an extra parameter of type \bluett{int}, which is just a tag.
            \end{itemize}
        \end{column}
    \end{columns}
\end{frame}

\subsection{Dereference and Arrow Operators}

\begin{frame}[fragile]{Dereference and Arrow Operators}
    \begin{cpp}
template <typename T, bool is_const>
class Slist_iterator {
 public:
  reference operator*() const {
    // return reference to the data
    // denoted by this iterator
  }
  pointer operator->() const {
    return std::addressof(operator*());
  }
};
    \end{cpp}
    \begin{itemize}
        \item To make \ttt{(*p).mem} equivalent to \ttt{p->mem}, \ttt{operator->} is always defined as this.
        \item Why do we use \ttt{std::addressof}? (\ttt{<memory>})
    \end{itemize}
\end{frame}

\subsection{Type Conversions}

\begin{frame}[fragile]{Type Conversion Operator}
    \begin{cpp}
class Rational {
 public:
  operator double() const {
    return (double)m_num / m_denom;
  }
};
Rational r = some_value();
double x = r;
    \end{cpp}
\end{frame}

\begin{frame}[fragile]{Conversion to \bluett{bool}}
    Recall that:
    \begin{cpp}
std::cin >> x;
if (std::cin)
  // input succeeded.
    \end{cpp}
    A conversion to \bluett{bool}?
    \begin{cpp}
class istream {
 public:
  operator bool() const {
    // ...
  }
};
    \end{cpp}
\end{frame}

\begin{frame}[fragile]{Surprise!}
    \begin{cpp}
class istream {
 public:
  operator bool() const {
    // ...
  }
};
    \end{cpp}
    With this conversion operator, the following code \textbf{compiles} happily!
    \begin{cpp}
int i;
std::cin << i; // ???
    \end{cpp}
\end{frame}

\begin{frame}[fragile]{\bluett{explicit} Conversion}
    \begin{cpp}
class istream {
 public:
  explicit operator bool() const {
    // ...
  }
};
    \end{cpp}
    Only allow explicit conversion to happen through this function.
    \begin{itemize}
        \item Using an expression as the operand of \bluett{operator\&\&}, \bluett{operator||} or \bluett{operator!},
        \item or in the condition part of \bluett{operator?:},
        \item or placing it in the condition part of \bluett{if, for, while, do-while} statements, \textbf{are also seen as explicit conversion to \bluett{bool}}.
    \end{itemize}
\end{frame}

\begin{frame}[fragile]{Be Careful with Ambiguity}
    \begin{cpp}
struct B;
struct A {
  A(const B &);
};
struct B {
  operator A() const;
};
B b;
A a(b); // ERROR! Which conversion?
    \end{cpp}
    \begin{itemize}
        \item Never define both a constructor \ttt{A::A(const B \&)} and a conversion \ttt{B::operator A() const}.
        \item Be careful with conversion to built-in types.
    \end{itemize}
\end{frame}

\begin{frame}[fragile]{Call Overloaded Operator Functions Directly}
    \begin{itemize}
        \item \ttt{a + b} may be equivalent to \ttt{operator+(a, b)} or \ttt{a.operator+(b)}.
        \item \ttt{++a} is equivalent to \ttt{a.operator++()},
        \item while \ttt{a++} should be \ttt{a.operator++(0)}. (Pass any integer you like.)
    \end{itemize}
    Remember to differentiate between members and non-members.
\end{frame}