\section{Templates}

\subsection{Function Templates}

\begin{frame}[fragile]{Define a Function Template}
    \begin{cpp}
int compare(int a, int b) {
  if (a < b) return -1;
  if (b < a) return 1;
  return 0;
}
int compare(double a, double b) {
  if (a < b) return -1;
  if (b < a) return 1;
  return 0;
}
int compare(const std::string &a, const std::string &b) {
  if (a < b) return -1;
  if (b < a) return 1;
  return 0;
}
    \end{cpp}
\end{frame}

\begin{frame}[fragile]{Define a Function Template}
    \begin{cpp}
template <typename T>
int compare(const T &a, const T &b) {
  if (a < b) return -1;
  if (b < a) return 1;
  return 0;
}
    \end{cpp}
    \begin{itemize}
        \item Type parameter: \ttt{T}.
        \item A template is a \textbf{guideline} for the compiler:
        \begin{itemize}
            \item When \ttt{compare(42, 53)} is called, \ttt{T} is deduced to be \bluett{int}, and the compiler generates a version of the function for \bluett{int}.
            \item When \ttt{compare(s1, s2)} is called on \ttt{std::string}s, \ttt{T} is deduced to be \ttt{std::string}.
            \item \textbf{Instantiation} of a template.
        \end{itemize}
    \end{itemize}
\end{frame}

\begin{frame}[fragile]{Type Parameter}
    \begin{cpp}
template <typename T, class U>
void fun(const T &, const U &);
    \end{cpp}
    \begin{itemize}
        \item \typename and \class here are \textbf{equivalent}: \ttt{U} doesn't have to be a class type.
    \end{itemize}
    \pause
    \begin{cpp}
fun(42, 3.14);  // T=int, U=double
std::string s = "Hello";
fun(s, 42);     // T=std::string, U=int
fun<int, int>(42, 3.14); // T=U=int, 3.14 converts to 3
    \end{cpp}
\end{frame}

\begin{frame}[fragile]{Requirements on Types}
    Is this good?
    \begin{cpp}
template <typename T>
int compare(T a, T b) {
  if (a < b) return -1;
  if (a > b) return 1;
  return 0;
}
    \end{cpp}
    \pause
    \begin{itemize}
        \item What if \ttt{T} is uncopyable, or the copying is too costly?
        \pause
        \item To make a class fit here, you need to define both \bluett{operator<} and \bluett{operator>}!
    \end{itemize}
    Template programs should try to minimize the number of requirements placed on the argument types.
\end{frame}

\begin{frame}[fragile]{Deduction Fails...}
    \begin{cpp}
template <typename T, typename U>
void fun(const U &x) {}
fun(42); // Error: What is T?
    \end{cpp}
    \pause
    You need to write it explicitly:
    \begin{cpp}
fun<double>(42); // Correct. T=double, U=int.
    \end{cpp}
\end{frame}

\subsection{Class Templates}

\begin{frame}[fragile]{Define a Class Template}
    \begin{itemize}
        \item HW5-1: \ttt{IntArray} \(\Longrightarrow\) \ttt{Array<T>}.
        \item Define member functions inside the class, and outside the class.
        \item Template arguments could be left out inside the class.
        \begin{columns}
            \begin{column}{0.5\textwidth}
                \begin{cpp}
// inside the class
Array<T> &operator=
    (const Array<T> &other) {
  // ...
}
                \end{cpp}
            \end{column}
            \begin{column}{0.5\textwidth}
                \begin{cpp}
// leave out <T>
Array &operator=
    (const Array &other) {
  // ...
}
                \end{cpp}
            \end{column}
        \end{columns}
    \end{itemize}
\end{frame}

\begin{frame}[fragile]{Class Templates}
    Template is a guideline for the compiler:
    \begin{itemize}
        \item When we use \ttt{Array<double>}, \ttt{T} becomes \bluett{double} and the compiler generates the code for \ttt{class Array<double>}.
        \item Classes instantiated from the same class template with different template arguments \textbf{have nothing to do with each other}. They are independnet classes.
        \begin{cpp}
template <typename T>
class Array {
  void fun(Array<double> &ad) {
    // Error whenever T is not double!
    auto x = ad.m_size; // m_size is private
  }
};
        \end{cpp}
    \end{itemize}
\end{frame}

\begin{frame}[fragile]{Templated Member Functions}
    \begin{cpp}
template <typename T>
class Array {
 public:
  Array(std::size_t n) : m_size(n), m_data(new T[n]{}) {}
  template <typename Iterator>
  Array(Iterator begin, Iterator end)
      : Array(std::distance(begin, end)) {
    std::copy(begin, end, m_data);
  }
};
    \end{cpp}
\end{frame}

\begin{frame}[fragile]{Templated Member Functions}
    Define it outside the class:
    \begin{cpp}
template <typename T>
template <typename Iterator>
Array<T>::Array(Iterator begin, Iterator end)
    : Array(std::distance(begin, end)) {
  std::copy(begin, end, m_data);
}
    \end{cpp}
    \begin{itemize}
        \item It should \textbf{NOT} be declared as
        \begin{cpp}
template <typename T, typename Iterator>
        \end{cpp}
    \end{itemize}
\end{frame}

\begin{frame}[fragile]{Templated Member Functions of Non-template Classes}
    \begin{cpp}
class Widget {
 public:
  template <typename Container>
  void add_this_to(Container &c) const {
    c.emplace_back(this);
  }
};
    \end{cpp}
\end{frame}

\begin{frame}[fragile]{Templated Member Functions of Non-template Classes}
    Define it outside the class:
    \begin{cpp}
class Widget {
 public:
  template <typename Container>
  void add_this_to(Container &) const;
};
template <typename Container>
void Widget::add_this_to(Container &c) const {
  c.emplace_back(this);
}
    \end{cpp}
\end{frame}

\begin{frame}[fragile]{Instantiation}
    \begin{itemize}
        \item For a class template, only when the class is used will the code for the class be generated.
        \item For a function template, only when the function is called will the code for the function be generated.
        \pause
        \item For a class template, a member function is compiled only when the function is called.
    \end{itemize}
\end{frame}

\begin{frame}[fragile]{Instantiation}
    \begin{cpp}
template <typename T>
class Array {
 public:
  void add_five() {
    for (std::size_t i = 0; i != m_size; ++i)
      m_data[i] += 5;
  }
};
Array<int> ai(10);
ai.add_five();              // OK
Array<std::string> as(10);
// Still OK. add_five is not compiled here.
    \end{cpp}
    \pause
    \begin{cpp}
as.add_five(); // Error.
    \end{cpp}
\end{frame}

\begin{frame}[fragile]{Templated Type Aliases}
    \begin{cpp}
template <typename T>
using pvec = std::shared_ptr<std::vector<T>>;

pvec<int> pvi; // pvi is a shared_ptr<vector<int>>
pvec<double> pvd; // pvd is a shared_ptr<vector<double>>
    \end{cpp}
    This is what \bluett{typedef} unable to do.
\end{frame}

\subsection{Template Specialization}

\begin{frame}[fragile]{Example}
    \begin{cpp}
template <typename T>
int compare(const T &a, const T &b) {
  if (a < b) return -1;
  if (b < a) return 1;
  return 0;
}
    \end{cpp}
    \pause
    \begin{cpp}
const char *cs1 = "hello", *cs2 = "world";
auto result = compare(cs1, cs2); // Oops!
    \end{cpp}
    \begin{itemize}
        \item \ttt{T} becomes \bluett{const char }\ttt{*}.
        \item Comparing two pointers instead of two strings!
    \end{itemize}
\end{frame}

\begin{frame}[fragile]{Specialization of Function Templates}
    \begin{cpp}
template <typename T>
int compare(const T &a, const T &b) {
  if (a < b) return -1;
  if (b < a) return 1;
  return 0;
}
template <>
int compare(const char *const &p1,
            const char *const &p2) {
  return std::strcmp(p1, p2);
}
    \end{cpp}
    \begin{itemize}
        \item Parameter types should match the corresponding types in the previously declared template.
        \begin{itemize}
            \item This is a specialization for \ttt{T = }\bluett{const char }\ttt{*}.
        \end{itemize}
    \end{itemize}
\end{frame}

\begin{frame}[fragile]{Specialization and Function Matching}
    \begin{itemize}
        \item Non-template \(>\) Template-specialization \(>\) Template.
    \end{itemize}
    \begin{cpp}
template <>
int compare(const char *const &p1,
            const char *const &p2) {
  std::cout << "template specialization" << std::endl;
  return std::strcmp(p1, p2);
}
int compare(const char *const &p1,
            const char *const &p2) {
  std::cout << "non-template" << std::endl;
  return std::strcmp(p1, p2);
}
const char *cs1 = "hello", *cs2 = "world";
auto result = compare(cs1, cs2); // "non-template"
    \end{cpp}
\end{frame}

\begin{frame}[fragile]{Specialization of Class Templates}
    If we want to make a specialized version for \ttt{Array<bool>}, just like \ttt{std::vector}...
    \begin{cpp}
template <typename T>
class Array { /* ... */ };

template <>
class Array<bool> {
  // Show your talented design for bools here.
};
    \end{cpp}
\end{frame}

\begin{frame}[fragile]{Partial Specialization}
    Recall that \ttt{std::vector} has another template parameter: \blue{the allocator}.
    \begin{cpp}
template <typename T, typename Alloc>
class vector { /* ... */ };

template <typename Alloc>
class vector<bool, Alloc> { /* ... */ };
    \end{cpp}
    \pause
    \begin{itemize}
        \item \ttt{vector<bool, Alloc>} is a specialization for \ttt{T=bool}, but it is still a template.
        \begin{itemize}
            \item \ttt{Alloc} is the remaining template parameter.
        \end{itemize}
    \end{itemize}
\end{frame}

\begin{frame}[fragile]{Partial Specialization}
    \ttt{std::unique\_ptr} has a partial specialization for arrays:
    \begin{cpp}
template <typename T>
class unique_ptr { /* ... */ };

template <typename T>
class unique_ptr<T[]> { /* ... */ };
    \end{cpp}
    \begin{itemize}
        \item \ttt{std::unique\_ptr<T[]>} does not provide \bluett{operator->}, but it provides \bluett{operator[]}.
        \item Used for managing dynamic arrays of unique ownership.
        \pause
        \item \textbf{Use STL containers} instead of \ttt{std::unique\_ptr<T[]>}, unless for special purposes!
    \end{itemize}
\end{frame}

\begin{frame}[fragile]{Partial Specialization}
    Partial specialization is \textbf{NOT allowed} for function templates:
    \begin{cpp}
template <typename T>
void swap(T &a, T &b) {
  T tmp(a);
  a = b;
  b = tmp;
}
template <typename T>
void swap<Array<T>>(Array<T> &a, Array<T> &b) { // Error!
  a.swap(b);
}
    \end{cpp}
\end{frame}

\subsection{Non-type Template Parameters}

\begin{frame}[fragile]{Integer Parameters}
    A new way of passing arrays:
    \begin{cpp}
template <unsigned N>
void print_array(const int (&arr)[N]) {
  for (unsigned i = 0; i != N; ++i)
    std::cout << arr[i] << " ";
}
    \end{cpp}
    \pause
    You may also use range-for:
    \begin{cpp}
template <unsigned N>
void print_array(const int (&arr)[N]) {
  for (auto x : arr)
    std::cout << x << " ";
}
    \end{cpp}
    \ttt{arr} is recognized as a \textbf{reference to array} here.
\end{frame}

\begin{frame}[fragile]{Integer Parameters}
    Recall the old way:
    \begin{cpp}
void print_array(const int *arr, unsigned N) {
  for (unsigned i = 0; i != N; ++i)
    std::cout << arr[i] << " ";
}
    \end{cpp}
    \ttt{arr} is simply recognized as a pointer here.
\end{frame}

\begin{frame}[fragile]{Integer Parameters}
    Generalization for value type:
    \begin{cpp}
template <typename T, unsigned N>
void print_array(const T (&arr)[N]) {
  for (const auto &x : arr)
    std::cout << x << " ";
}
    \end{cpp}
    \begin{itemize}
        \item Be careful with unknown types! (use \bluett{const auto }\ttt{\&} to avoid copying)
    \end{itemize}
\end{frame}

\begin{frame}[fragile]{Integer Parameters}
    \begin{cpp}
template <typename T, bool is_const>
class Slist_iterator {
  // When is_const is true, it is a const_iterator.
  // Otherwise, it is an iterator.
};

template <typename T>
class Slist {
 public:
  using iterator = Slist_iterator<T, false>;
  using const_iterator = Slist_iterator<T, true>;
  // ...
};
    \end{cpp}
\end{frame}

\begin{frame}[fragile]{Constant Expression}
    \begin{itemize}
        \item Anything used as a template argument must be \textbf{known at compile-time}. \red{(Why?)}
        \pause
    \end{itemize}
    \begin{cpp}
int n;
std::cin >> n;
std::array<int, n> arr; // Error! n is runtime-determined.
    \end{cpp}
    \pause
    \begin{cpp}
int maxn = 1000;
std::array<int, maxn> arr; // Error! maxn is not a constant expression
    \end{cpp}
\end{frame}